\documentclass{article}
\usepackage[utf8x]{inputenc}
\usepackage{eurosym}
\usepackage{babel}
\usepackage{hyperref}
\hypersetup{
    colorlinks=true,
    linkcolor=blue,
    filecolor=magenta,      
    urlcolor=cyan,
}
\usepackage[left=1cm, right=1cm, top=1cm, bottom=2cm]{geometry}
\makeatletter
\renewcommand{\maketitle}
{
	\bgroup\setlength{\parindent}{0pt}
	\begin{flushleft}\textbf{\@author}\end{flushleft}
	\begin{center}\huge\@title\end{center}
}

\newcommand{\lesson}[3]
{
	\item \href{https://openclassrooms.com/courses/#1}{#2} #3 \%
}

\begin{document}

	\author{Benjamin Loison\\
	Né le 23 septembre 2000\\
	1 rue de quelque part 00000 Ville\\
	06 12 34 56 78\\
	benjamin.loison@lemnoslife.com}
	\title{\vspace*{2cm}Curriculum Vitae (version GitHub)\\
	\Large juin 2019\\
	\normalsize \textit{\\Ce document utilise des liens hypertextes (URL), à utiliser avec un logiciel adéquat afin de le visualiser}}
	\maketitle

	\vspace*{1cm}
	\def\contentsname{\empty}
	\tableofcontents

	\vspace*{2cm}

	\section{Langages de programmation déjà utilisés}

		\begin{enumerate}

			\item Extrêmement familier (utilisation quotidienne) avec:\\
				C++ (avec les bibliothèques SDL, OpenGL, OpenAL, Curl, OpenCV, NanoSVG, Cereal...), C, Java, Python, Caml (OCaml et CamlLight), PhP, \href{https://www.wolfram.com/language/elementary-introduction/2nd-ed/}{Wolfram} (Mathematica), HTML5, JavaScript, Casio BASIC, Texas Instrument BASIC, \LaTeX, Batch, Bash, \href{https://community.bistudio.com/wiki/SQF\_syntax}{SQF}, Blockly, Scratch et Pseudocode
			\item Très familier avec: JavaEE, SQL, CSS3, jQuery, AngularJS, Node.js, C\#, Ruby (+ on Rails) et Objective-C
			\item Familier avec: R, Scala, Octave, UML, Perl, Assembly, Swift, OpenCL, Cuda et Lua
			\item Moins familier avec: Binaire, Fortran, Lisp et Julia 

		\end{enumerate}

	\section{Systèmes d'exploitation déjà utilisés}

		\begin{enumerate}

			\item Windows:
				\subitem Extrêmement familier: Windows 10, Windows 8.1 et Windows 7
				\subitem Très familier: Windows Vista et Windows XP

			\item Apple:
				\subitem Très familier: iOS
				\subitem Familier: Mac OS High Sierra

			\item Linux:
				\subitem Extrêmement familier: Android, Ubuntu et Debian
				\subitem Très familier: Kali et Mint

		\end{enumerate}

	\section{Principales réalisations}

		Beaucoup des projets suivants sont disponibles sur \href{https://github.com/Benjamin-Loison?tab=repositories}{mon GitHub: https://github.com/Benjamin-Loison?tab=repositories}

		\begin{enumerate}

			\item \href{https://github.com/Benjamin-Loison/LemnosLife}{Mon jeu vidéo LemnosLife} codé en C++ (avec OpenGL, SDL, OpenAL, Cereal, NanoSVG):

				\subitem \href{https://github.com/Benjamin-Loison/LemnosLife/tree/master/ServerBridge}{Utilisation d'un serveur relais hébergé chez OVH afin d'empêcher tout DDOS sur le serveur de jeu terminal hébergé à domicile}
				\subitem \href{https://github.com/Benjamin-Loison/LemnosLife/tree/master/Installation}{Utilisation d'un système d'installation et de mises à jours automatiques et optimisées}
				\subitem \href{https://github.com/Benjamin-Loison/LemnosLife/tree/master/Client/LemnosLife/Map}{Gestion des données compressées de 900 km² (île Lemnos en Grèce) du monde réel.}
				\subitem \href{https://github.com/Benjamin-Loison/LemnosLife/tree/master/LaTex/Theory}{Gestion de la physique et des modèles mathématiques pour les collisions, les avions de chasse, les armes, les véhicules, la gravité.}
				\subitem Gestion du chat vocal, du \href{https://github.com/Benjamin-Loison/LemnosLife/tree/master/Client/LemnosLife/Render}{moteur graphique} et de \href{https://github.com/Benjamin-Loison/LemnosLife/blob/master/Client/Network/Main/client.h}{la partie réseau}, avec \href{https://github.com/Benjamin-Loison/Lot-of-Java-projects/tree/master/ServeurAuth}{l'authentification}...
				\subitem Gestion de l'ensemble du projet avec un \href{https://github.com/Benjamin-Loison/LemnosLife/tree/master/Panel}{panel pour programmer} fait par moi-même et \href{https://docs.google.com/document/d/1rnYh5uGzhmrSZ34OW5-3V3XZjxATDYzPypT43C2trjE}{d'un cahier des charges}.
				\subitem Gestion de l'ensemble des programmeurs et des graphistes.

			\item \href{https://github.com/Benjamin-Loison/C--projects/blob/master/main.cpp}{Expériences en cryptographie avec l'algorithme RSA} et \href{https://github.com/Benjamin-Loison/Lot-of-Java-projects/tree/master/Hash\%20password\%20database}{l'algorithme de cryptage symétrique Blowfish}
			\item \href{https://github.com/Benjamin-Loison/Lot-of-Java-projects/tree/master/My4GGSM}{My4GSM}: Génération de 4G avec le réseau GSM (SMS/MMS) - Android
			\item \href{https://github.com/Benjamin-Loison/AltisCraft.fr}{Projets Minecraft: AltisCraft.fr} (plus de 40 000 lignes de code de Java et plus de 85 948 personnes y ont joué), mini-jeux, nombreux \href{https://github.com/Benjamin-Loison/Lot-of-Java-projects/blob/master/Minecraft\%20mods\%20and\%20plugins.zip}{mods} et \href{https://github.com/Benjamin-Loison/Azziz-Plugin-MC-Mini-Games}{plugins}...
			\item \href{https://github.com/Benjamin-Loison/Verificateur-de-carte-bancaire}{Vérificateur de carte bleue (Python)}
			\item \href{https://github.com/Benjamin-Loison/Wolfram-Collision}{Modèles de collisions avec Wolfram}
			\item Fractales:
				\subitem \href{https://github.com/Benjamin-Loison/Koch-snowflake}{Le flocon de Koch avec explication de la démarche/code en 3 épisodes sur YouTube (voir GitHub)} (Casio/TI BASIC, Python et C++)
				\subitem \href{https://github.com/Benjamin-Loison/Conway-game-of-life}{Le jeu de la vie de Conway} (Casio/TI BASIC, Python et C++)
				\subitem L'ensemble de Mandelbrot (Casio/\href{https://github.com/Benjamin-Loison/BASIC-algorithms-calculators-/tree/master/MANDELBR}{TI} BASIC et Python)
				\subitem \href{https://github.com/Benjamin-Loison/Sierpinski-s-triangle}{Le triangle de Sierpinski (Python)}
			\item Automate cellulaire:
				\subitem \href{https://github.com/Benjamin-Loison/Langton-s-ant}{La fourmi de Langton (Python et C++)}
			\item \href{https://github.com/Benjamin-Loison/BASIC-algorithms-calculators-}{Algorithmes très variés (fractales, exercices, test mathématiques, jeux...) sur Casio 35+e ou Texas Instruments 83 Premium}
			\item \href{https://github.com/Benjamin-Loison/ArduinoFun}{Réalisations électriques avec Arduino} et Raspberry pi
			\item \href{https://github.com/Benjamin-Loison/FreeLance}{Expérience en tant que free lance sur la plateforme HopWork (Malt) sur quelques projets avec interface web et applications hybrides iOS et Android}
			\item \href{https://github.com/Benjamin-Loison/Travaux-Personnel-Encadr-s-TPE---BAC-}{Site internet pour mon Travail Personnel Encadré (TPE) pour le BAC sur le Temps, codé à la main par moi-même pour exposer notre travail.}
			\item Et plus d'une centaine de projets qui n'attendent que mon temps libre...
			
		\end{enumerate}

	\section{Formation}

	\begin{enumerate}

		\item \href{https://github.com/Benjamin-Loison/Memos}{Mes mémos témoignent de la variété d'outils que je sais utiliser avec les différents langages de programmation et les différents systèmes d'exploitation}
		\item 2018-2019 \href{https://github.com/Benjamin-Loison/MPSI1}{Classe préparatoire en MPSI}:
			\subitem \href{https://github.com/Benjamin-Loison/MPSI1/tree/master/IPT\%20TPs}{En Informatique Pour Tous (IPT) (Python)}
			\subitem \href{https://github.com/Benjamin-Loison/MPSI1/tree/master/OCaml\%20TPs}{En option informatique (OCaml)}
			\subitem \href{https://github.com/Benjamin-Loison/MPSI1/tree/master/TIPE}{Pour mon Travail d'Initiative Personnelle Encadré (TIPE) sur le thème "océan" avec une approche de comptage et reconnaissance de poissons sur une image avec des réseaux de neurones (Python, OpenCV, C++, OpenCL)}
			\subitem \href{https://github.com/Benjamin-Loison/MPSI1/blob/master/Physic-chemistry.zip}{En physique-chimie: utilisation de \LaTeX\ pour tous les compte-rendus de travaux pratiques}
			\subitem \href{https://github.com/Benjamin-Loison/MPSI1/tree/master/Mathematics}{En mathématiques: utilisation de Mathematica pour vérifier les résultats
			et pour certains documents que j'ai réalisés en \LaTeX\ pour les devoirs maisons et pour le TIPE}.
		\item 2017-2018 Baccalauréat série S, spécialité mathématiques, mention Bien
		\item 2015-2016 Seconde générale avec \href{https://github.com/Benjamin-Loison/ISN-2-nde-French-educationnal-system-}{option Information et Sciences du Numérique (ISN) orienté en Python et en HTML5/CSS3}
		
	\end{enumerate}

	\section{Evénements}

		\begin{enumerate}

			\item Participation aux pépinières de mathématiques de Premières (2016-2017) à Ville.
			\item 2 semaines à l'école d'été de Wolfram à Oxford en juillet 2017 avec l'utilisation de Mathematica pour résoudre des problèmes mathématiques (et découvrir de nouvelles notions de mathématiques). J'y ai participé avec d'autres lauréats du concours Algoréa de France-IOI et lauréats des olympiades françaises de mathématiques.
			\item Compétitions entre développeurs: France-IOI (Algoréa 2016) - 33\textsuperscript{ème} de France, BattleDev - 479\textsuperscript{ème} de France en novembre 2016 et 552\textsuperscript{ème} de France en novembre 2018.

		\end{enumerate}

	\section{Compétences matérielles}

		\begin{enumerate}

			\item Montage d'ordinateurs:
				\subitem Ordinateur actuellement utilisé:
					CPU: i9 7920X (4.9 Ghz)
					GPU: Titan Xp
					RAM: 16 Go (4.2 Ghz)
					SSD: 1 026 G (avec 800 branché en PCI), 2 To de HDD (USB) et 3 To de HDD (USB afin de procéder à des sauvegardes journalières (et de les conserver pendant des mois grâce à leurs poids minimales) avec rsnapshot à l'aide de cygwin)
					Ecrans: 4K et 2 de 1920 x 1080
					Ping: 0 ms
					Envoi et téléchargement: 500 mb/s
				\subitem Serveur à domicile: i7, double SSD: un pour gérer en temps réel mes services (mon serveur de jeu, mon site web, mon serveur vocal) et un autre pour émuler un Windows et un Debian pour compiler mon jeu, avec deux HDD qui sauvegardent les 2 SSD toutes les heures, conservant plusieurs mois de sauvegarde.
				\subitem Un ordinateur d'essai de montage: AMD Athlon 2, HDD...
			\item Ordinateur portable ASUS (Republic of Gamer)

		\end{enumerate}

	\section{Compétences de piratage}
	
		Dans une optique de compréhension des dangers du numérique, j'ai déjà testé, dans un cadre de sécurité strict, les techniques suivantes sur moi-même:
		
		\begin{enumerate}
		
			\item \href{https://github.com/Benjamin-Loison/Lot-of-Java-projects/tree/master/BruteForce\%20HTTP\%20Auth}{Connaissance des limites de l'énumération exhaustive}
			\item Connaissance de la création et l'utilisation d'un cheval de Troie (Trojan, \href{https://github.com/Benjamin-Loison/Lot-of-Java-projects/tree/master/TrojanServer}{Java}/\href{https://github.com/Benjamin-Loison/Trojan}{C++}, UDP/TCP)
			\item Matériels adéquats à une attaque "man in the middle" (MITM): clé USB wifi très puissante, attaque physique avec charge utile (payload): USB Rubber Ducky et destruction: USB Killer
		
		\end{enumerate}

	\section{Culture informatique}

		\begin{enumerate}

			\item Visionnage de toutes les vidéos des chaînes YouTube suivantes:
				\subitem \href{https://www.youtube.com/channel/UCOLNDRtP8tUK5i-6jLO0r4Q}{Guillaume Slash}, \href{https://www.youtube.com/channel/UCYnvxJ-PKiGXo_tYXpWAC-w}{Micode}, \href{https://www.youtube.com/user/dexsilicium}{Deus Ex Silicium}, \href{https://www.youtube.com/channel/UCU8_onYYhNxwGUB3hA34C4Q}{FrenchHardware}, \href{https://www.youtube.com/user/jojol67}{Jojol}, \href{}{} \href{https://www.youtube.com/user/TechNewsTests}{TechNews\&Tests}, \href{https://www.youtube.com/user/Computerphile}{computerphile}, \href{https://www.youtube.com/watch?v=aircAruvnKk&list=PLZHQObOWTQDNU6R1_67000Dx_ZCJB-3pi}{3Blue1Brown}, \href{https://www.youtube.com/channel/UCRhyS_ylPQ5GWBl1lK92ftA}{Léo - TechMaker}, \href{https://www.youtube.com/watch?v=A5dl7XRnmM4&list=PL5TvtEevd4IbBjsZe94pdU8r98DVr71rm}{Lucas Willems}, \href{https://www.youtube.com/channel/UC0JUkXAVVA4qWH1BQRs5N3A}{PAUSE PROCESS}, \href{https://www.youtube.com/watch?v=IcrBqCFLHIY}{Veritasium}, \href{https://www.youtube.com/watch?v=2aCS5mEeiwg}{l'Esprit Sorcier}, \href{https://www.youtube.com/watch?v=G5s4-Kak49o}{Vsauce}, \href{https://www.youtube.com/watch?v=pXKMIJ8GOPg}{e-penser}, \href{https://www.youtube.com/channel/UCEKlKgrnLlkSzG-MKMYoSMA}{Thomas Cyrix}, \href{https://www.youtube.com/watch?v=xuBzQ38DNhE}{ScienceEtonnante}, \href{https://www.youtube.com/watch?v=I5dUx-l-HmE}{Mathieu Nebra} et \href{https://www.youtube.com/watch?v=DrjkjPVf7Bw}{Science4All} qui sont majoritairement des vulgarisateurs d'informatiques ou qui décrivent leur ressenti vis-à-vis de nouvelles technologies. Certaines comme PAUSE PROCESS, Science Etonnante, computerphile, 3Blue1Brown, Deus Ex Silicium, Mathieu Nebra et Science4All m'ont totalement fasciné par leur qualité.
			\item Lecture:
				\subitem Pour les nuls: Programmer avec Python en s'amusant
				\subitem Gallimard: Histoire de la révolution numérique 
				\subitem Eyrolles: Programmation système en C sous Linux
				\subitem Comment ça marche.net: Tout sur le Hardware PC 2ème édition
				\subitem Hackable magazine: 6 jours pour débuter facilement avec Arduino
				\subitem Pour les nuls: PHP \& MySQL
				\subitem Stephen Wolfram: Une introduction élémentaire à Wolfram Language, deuxième édition (An elementary introduction to the Wolfram Language, second edition)
		
		\item Cours principaux suivis sur \href{https://openclassrooms.com/fr/membres/benjaminloison}{OpenClassrooms} (liste complète disponible \href{https://github.com/Benjamin-Loison/Curriculum-vitae/blob/master/Lessons\%20followed\%20on\%20OpenClassrooms.pdf}{ici})
	
		\begin{enumerate}

			\lesson{1362801}{La cryptographie asymétrique : RSA}{100}
			\lesson{1617396}{Rédigez des documents de qualité avec LaTeX}{55}
			\lesson{2976551}{Develop your venture (Développez votre entreprise)}{94}
			\lesson{2709621}{Decode the entrepreneur's DNA (Décoder l'ADN de l'entrepreneur)}{92}
			\lesson{1129721}{Le typage : présentation thématique et historique}{100}
			\lesson{1818146}{Apprenez à programmer en Caml}{100}
			\lesson{235344}{Apprenez à programmer en Python}{95}
			\lesson{26832}{Apprenez à programmer en Java}{87}
			\lesson{1170641}{[C++] Les conversions de types}{100}
			\lesson{1894236}{Programmez avec le langage C++}{92}
						 
		\end{enumerate}
		
		\item Cours en anglais sur le machine learning suivi sur \href{https://www.coursera.org/learn/machine-learning?}{Coursera}

	\end{enumerate}

	\section{Moyens de communication}

		\begin{enumerate}

			\item Email: benjamin.loison@lemnoslife.com
			\item GitHub: \href{https://github.com/Benjamin-Loison}{Benjamin Loison}
			\item France-IOI: \href{http://france-ioi.org/user/perso.php?sLogin=benjamin_loison}{benjamin\_loison}
			\item TeamSpeak: ts.lemnoslife.com
			\item Site web: \href{https://lemnoslife.com}{LemnosLife.com}
			\item Chaîne YouTube: \href{https://youtube.com/c/BenjaminLoison}{Benjamin Loison} et \href{https://www.youtube.com/channel/UCt5USYpzzMCYhkirVQGHwKQ}{chaîne secondaire}
			\item StackOverflow: \href{https://stackoverflow.com/users/7123660/benjamin-loison}{Benjamin Loison}

		\end{enumerate}
		
	\section{Code source de ce Curriculum Vitae}
	
		\url{https://github.com/Benjamin-Loison/Curriculum-vitae}

\end{document}
